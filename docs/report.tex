\documentclass[10pt,letterpaper]{article}
\usepackage[utf8]{inputenc}
\usepackage{amsmath}
\usepackage{amsfonts}
\usepackage{amssymb}
\usepackage[left=2cm,right=2cm,top=2cm,bottom=2cm]{geometry}
\usepackage{hyperref}
\hypersetup{
    colorlinks=true,
    linkcolor=blue,
    filecolor=magenta,
    urlcolor=cyan,
}
\author{Matthew Burket and Joel May}
\title{Programming Assignment 2 Report}

\begin{document}
\maketitle

\section{Data Structures}
\subsection{Q}
We used an \texttt{ArrayDeque<String>} for $Q$. We needed a simple queue that lets us enqueue and dequeue items. In Java, the \texttt{ArrayDeque} is one of the most basic queues. It was a trivial choice to make the queue hold \texttt{String}s for the nodes. There was no obvious contender on that choice.
\subsection{Visited}
We used a \texttt{HashMap<String, List<String>>} for $visited$. The operations we need are inserting new URLs, checking for the existence of URLs as keys, and iterating through all keys with their values. The \texttt{HashMap} gives us $O(1)$ time complexity for those operations (assuming hash functions of $O(1)$). The \texttt{HashMap} implements what we need and does the operations quickly.

In \texttt{NetworkInfluence}, we used a \texttt{HashMap<String, String>} to serve a dual purpose. First, the key served the same purpose as the \texttt{HashSet<String> visited} above. Second, this map stored the "backlink" to allow locating the path used to traverse the graph. The \texttt{HashMap} works quite well, because the important operations (contains key and get value by key) are all $O(1)$.
\section{Number of edges and vertices}
Edges: 1,147; Nodes: 100
\section{Top 10 most influential nodes}
\subsection{computed by mostInfluentialDegree}
\begin{itemize}
\item /wiki/Embedded\_system
\item /wiki/Peripheral
\item /wiki/Information\_Age
\item /wiki/Computer\_graphics\_(computer\_science)
\item /wiki/Computational\_complexity\_theory
\item /wiki/Mathematical\_logic
\item /wiki/Computing\_technology
\item /wiki/Software\_development
\item /wiki/Programming\_paradigm
\item /wiki/Computer\_Science
\end{itemize}
\subsection{computed bymostInfluentialModular}
\begin{itemize}
\item /wiki/Peripheral
\item /wiki/Lois\_Haibt
\item /wiki/Information\_Age
\item /wiki/Computational\_complexity\_theory
\item /wiki/Mathematical\_logic
\item /wiki/Software\_development
\item /wiki/Programming\_paradigm
\item /wiki/Computer\_graphics\_(computer\_science)
\item /wiki/Computing\_technology
\item /wiki/Computer\_Science
\end{itemize}
\subsection{computed by mostInfluentialSubModular}
\begin{itemize}
\item /wiki/Computer\_Science
\item /wiki/Video\_post-processing
\item /wiki/Natural\_science
\item /wiki/Legged\_Squad\_Support\_System
\item /wiki/Historical\_Dictionary\_of\_Switzerland
\item /wiki/Interpreter\_(computing)
\item /wiki/Neolithic\_Revolution
\item /wiki/Glossary\_of\_geology
\item /wiki/Word\_processor
\item /wiki/Glossary\_of\_biology
\end{itemize}
\section{Pseudo Code}
\subsection{Constructor}
\begin{verbatim}
function NetworkInfluence(file)
    vertexCount = file.readInt()
    while file.hasMoreLines()
        line = file.readLine().split(" ")
        // edges is a HashMap<node, List<node>>.
        // Either initialize the list or append to the list if it exists.
        edges.createOrAppendToList(line[0], line[1])

        // nodes is a hashmap, so no duplicates are created when nodes are re-encountered
        nodes.add(line[0])
        nodes.add(line[1])
\end{verbatim}
\subsection{\texttt{outDegree}}
\begin{verbatim}
function outDegree(v)
    if edges.containKey(v)
        return edges.get(v).size()
    return 0
\end{verbatim}
\subsection{\texttt{shortestPath}}
\begin{verbatim}
function shortestPath(u, v)
    Queue toVisit
    HashMap<String, String> parent

    toVisit.add(u)
    parent.put(u, null)
    do
        currentNode = toVisit.remove()
        if currentNode == v
            backNode = currentNode
            List path
            do
                path.addFront(backNode)
                backNode = parent.get(backend)
            while !backNode == u
            return path
        foreach child in edges.get(currentNode)
            if !parent.containsKey(child)
                // Store the backlink
                parent.put(child, currentNode)
                toVisit.add(child)
    while !toVisit.empty()
    // No path
    return {}
\end{verbatim}
\subsection{\texttt{distance(String u, String v)}}
\begin{verbatim}
function distance(String u, String v) 
    return shortestPath(u,v).size() - 1
\end{verbatim}
\subsection{\texttt{distance(ArrayList<String> s, String v)}}
\begin{verbatim}
function distance(ArrayList<String> s, String v) 
    int minDistance = Int.MAX_VALUE
    foreach node in s
        int thisDist = shortestPath(u,v)
        if thisDist != -1 && thisDist < minDistance
            minDistance = thisDist
    if thisDist == Int.MAX_VALUE
        return -1
    return minDist
\end{verbatim}
\subsection{\texttt{influence(String s)}}
\begin{verbatim}
function influence(s)
    HashMap<Integer, Integer> tieredCounts = tieredBfs(s);
    sum = 0

    for depthAndCount in tieredCounts
        sum += 1 / 2^depthAndCount.key * depthAndCount.value

    return sum
\end{verbatim}
\subsection{\texttt{influence(ArrayList<String> s)}}
\begin{verbatim}
function influence(ArrayList<String> s)
    HashMap<Integer, Integer> tieredCounts = tieredBfs(s);
    sum = 0

    for depthAndCount in tieredCounts
        sum += 1 / 2^depthAndCount.key * depthAndCount.value

    return sum
\end{verbatim}
\subsection{\texttt{mostInfluentialDegree(int k)}}
\begin{verbatim}
function mostInfluentialDegree(int k)
    TreeSet<Map.Entry<String, Integer>> topK = new TreeSet<>(new NodeInfluenceComparator<>());
    
    ArrayList<String> ret = new ArrayList<>(k);
    
    for node in nodes
        topK.add(new Map.Entry(node, outDegree(node))
        
        while topK.size > k
            topK.remove(topK.first)
    
    for nodeAndInfluence in topK
        ret.add(nodeAndInfluence.key)
    
    return ret
\end{verbatim}
\subsection{\texttt{mostInfluentialModular(int k)}}
\begin{verbatim}
function mostInfluentialModular(int k)
    TreeSet<Map.Entry<String, Integer>> topK = new TreeSet<>(new NodeInfluenceComparator<>());
    
    ArrayList<String> ret = new ArrayList<>(k);
    
    for node in nodes
        topK.add(new Map.Entry(node, influence(node))
        
        while topK.size > k
            topK.remove(topK.first)
    
    for nodeAndInfluence in topK
        ret.add(nodeAndInfluence.key)
    
    return ret
\end{verbatim}
\subsection{\texttt{mostInfluentialSubModular(int k)}}
\begin{verbatim}
function mostInfluentialSubModular(int k)
    ArrayList<String> S = new ArrayList()
    HashSet<String> SHashSet = new HashSet<>()
    
    for i = 0 to k - 1
         String topNewNode = null;
         float topNewNodeInf = Float.NEGATIVE_INFINITY;
         int lastIndex = S.size();
         
         s.add(null)
         
         for node in nodes
             if not SHashSet.contains(node)
                S.set(lastIndex, node)
                float testInfluence = influence(S);
                if testInfluence > topNewNodeInf
                   topNewNode = node;
                   topNewNodeInf = testInfluence;
        
        S.set(lastIndex, topNewNode)
        SHashSet.add(topNewNode)
       
    return S
\end{verbatim}
\subsection{\texttt{tieredBfs(String startingNode)}}
This method is private, however it is used two places in public methods. Thus,
it is placed here for brevity in the public methods.
\begin{verbatim}
function tieredBfs(String startingNode)
    return tieredBfs({startingNode});
\end{verbatim}
\subsection{\texttt{tieredBfs(Iterable<String> startingNodes)}}
This method is private, however it is used two places in public methods. Thus,
it is placed here for brevity in the public methods.
\begin{verbatim}
function tieredBfs(Iterable<String> startingNodes)
    HashMap<Int, Int> ret = new TreeMap<>();
    ArrayDeque<Map.Entry<String, Integer>> toVisit = new ArrayDeque<>()
    HashSet<String> visited = new HashSet<>()

    for startingNode in startingNodes
        toVisit.add(new Map.Entry<>(startingNode, 0))
        visited.add(startingNode)

    do
        Map.Entry<String, Integer> currentNode = toVisit.remove()
        int newCount = ret.getOrDefault(currentNode.getValue(), 0) + 1
        ret.put(currentNode.value, newCount

        if edges.containsKey(currentNode.getKey())
            for dst in edges.get(currentNode.getKey())
                if not visited.contains(dst)
                    toVisit.add(new Map.Entry(dst, currentNode.value + 1)
                    visited.add(dst)
     while not toVisit.isEmpty()

     return ret
\end{verbatim}
\section{Run-time}
\subsection{Constructor}
It iterates through each line (edge) of the input file. So the main loop is $O(|Edges|)$. \texttt{edges} and \texttt{nodes} are hashed collections, which means the add operations take $O(1)$. All other operations in the constructor are constant or bounded by the URL length. The practical run time is thus $O(|Edges|)$.
\subsection{\texttt{outDegree}}
We assume that perfect hash function exists, thus $O(1)$ is need for the \texttt{containKey(v)} method and the run-time for \texttt{get(v)} is also $O(1)$. The call for getting the size of the \texttt{ArrayList} of the key is also $O(1)$.
Thus the run time is $O(1)$.
\subsection{\texttt{shortestPath}}
The queue only receives each node at most once, so the main loop is bounded by $O(|Nodes|)$. The queue and hashmap operations used (add, put, remove, isEmpty, and containsKey) are $O(1)$. In the worst case, each edge is visited in the inner for loop, which is $O(|Edges|)$.

The if-statement in the main loop only executes at most once (ends with return), so it is added to the run time of the nested loops. It consists of navigating a subset of edges, so it is $O(|Edges|)$. Adding to the linked list takes $O(1)$. Navigating the hashmap of back edges takes $O(1)$ for each lookup. So the if-statement is $O(|Edges|)$.

This comes down to $O(|Edges|*|Nodes|)+O(|Edges|)$. Simplifies to $O(|Edges|*|Nodes|)$.
\subsection{\texttt{tieredBfs} (private helper method)}
The queue only receives each node at most once, so the main loop is bounded by $O(|Nodes|)$. The queue and hashset operations used (add, remove, isEmpty, and contains) are $O(1)$. In the worst case, each edge is visited in the inner for loop, which is $O(|Edges|)$. The \texttt{ret} hashmap get and put operations takes $O(1)$. So the run time is $O(|Nodes|*|Edges|)$.
\subsection{\texttt{influence(String s)}}
First the method calls tieredBfs(s) which takes $O(|Edges| * |Node|)$\\
For each entry in the \texttt{HashMap}, which worst case would be $O(|Edges|)$, we do the following calculations.

\begin{verbatim}
Math.pow(2, depthAndCount.getKey()) 
\end{verbatim}
Which is $O(1)$ as JVM has some efficient ways of doing \texttt{Math.pow}, \href{https://stackoverflow.com/a/32419139/749721}{source}.
Then we divide $1$ by \texttt{Math.pow(2, depthAndCount.getKey())} which will
take $O(1)$. Then we times that value by \texttt{depthAndCount.getValue()} which should also take $O(1)$. Finally, we this value add to the value of \texttt{sum} which also takes $O(1)$.Which gives the overall run-time of $O(|Edges| * |Node|$.
\subsection{\texttt{influence(ArrayList<String> s)}}
Very similar to \texttt{influence(String s)} as code paths are basically the same.

First the method calls tieredBfs(s) which takes $O(|Edges| * |Node|)$\\
For each entry in the \texttt{HashMap}, which worst case would be $O(|Edges|)$, we do the following calculations.

\begin{verbatim}
Math.pow(2, depthAndCount.getKey()) 
\end{verbatim}
Which is $O(1)$ as JVM has some efficient ways of doing \texttt{Math.pow}, \href{https://stackoverflow.com/a/32419139/749721}{source}.
Then we divide $1$ by \texttt{Math.pow(2, depthAndCount.getKey())} which will
take $O(1)$. Then we times that value by \texttt{depthAndCount.getValue()} which should also take $O(1)$. Finally, we this value add to the value of \texttt{sum} which also takes $O(1)$.Which gives the overall run-time of $O(|Edges| * |Node|$.
\subsection{\texttt{mostInfluentialDegree}}
The main for loop is obviously $O(|Nodes|)$. Adding to and deleting from \texttt{topK} take $O(log k)$ each. Calculating the out degree was decided to be $O(1)$ as stated above. The last foreach, outside of the main loop, is $O(k)$. So we have $O(|Nodes|*log k)+O(k)$. When $|Nodes|<k$, $k=|Nodes|$, and the first term dominates. Thus the run time is $O(|Nodes|*log k)$
\subsection{\texttt{mostInfluentialModular}}
The main for loop is obviously $O(|Nodes|)$. Adding to and deleting from \texttt{topK} take $O(log k)$ each. Calculating the influence takes $O(|Nodes|*|Edges|)$, as stated above. The last foreach, outside of the main loop, is $O(k)$. So the run time is $O(|Nodes|^2*|Edges|)$.
\subsection{\texttt{mostInfluentialSubModular}}
The outer loop is $O(k)$. The arraylist operations (set and add) take $O(1)$. The hashset operations (contains and add) take $O(1)$. The inner for-loop is $O(|Nodes|)$. Calculating the influence takes $O(|Nodes|*|Edges|)$, as stated above. Thus the run time is $O(|Nodes|^2*|Edges|*k)$.
\subsection{\texttt{distance(String u, String v)}}
This method calls $shortestPath(u, v)$ which as described above takes $O(|Edges| * |Nodes|)$. Then 1 is subtracted from it, which takes $O(1)$. Thus giving us a run-time of $O(|Edges| * |Nodes|)$.
\end{document}